
\documentclass[12pt]{article}
\usepackage{jmiremu}

\title{An AI-Augmented, EHR-Centered Framework for the Full Medical Encounter}
\author[1]{Ivan Ogasawara}
\affil[1]{Independent Researcher, IGDORE}
\date{November 2025}

\begin{document}
\maketitle

\begin{structuredabstract}
\item[Background] Clinical care spans screening, diagnosis, treatment,
prescription, and follow-up. EHRs improved access but remain fragmented.
\item[Objective] Present a modular, AI-augmented framework that supports
the full encounter and multidisciplinary collaboration.
\item[Methods] Mixed-methods design with synthetic vignettes and clinician
evaluation, comparing standard EHR versus the proposed framework.
\item[Results] To be completed after evaluation.
\item[Conclusions] The framework emphasizes continuity, extensibility,
and collaboration, laying groundwork for real-world studies.
\end{structuredabstract}
\keywordsline{clinical decision support; electronic health records; artificial
intelligence; multimodal data; human--AI collaboration}

\section{Introduction}
High-quality care requires continuous, context-rich information across the
patient journey. While EHRs improved access, real-world workflows remain
fragmented \cite{lin2022jamia,cdc_nehrs_2021}. AI—including LLMs and
retrieval-augmented systems—can unify information flow. We introduce a
modular framework that supports all encounter phases and enables
multidisciplinary collaboration.

\section{Methods}
\subsection{Study Design}
Mixed-methods exploratory study combining design science and early evaluation
with clinicians using synthetic cases.

\subsection{Framework Development}
Requirements were synthesized from literature and informal clinician input.
The architecture models five phases (screening to follow-up) backed by modular
AI plugins and a unified patient record. A mid-fidelity prototype demonstrates
workflow feasibility.

\subsection{Early Evaluation}
Participants: 6--12 clinicians from multiple specialties. Materials: two
standardized synthetic vignettes (acute and chronic complex). Procedure:
counterbalanced sessions comparing standard EHR and framework-assisted
workflows; think-aloud and post-task interviews.

\subsection{Measures and Analysis}
Quantitative: perceived usefulness, trust, decision support, explainability,
information continuity, workflow fit, and optional workload. Paired tests
(e.g., Wilcoxon) and effect sizes. Qualitative: inductive thematic analysis
of interviews; triangulation of findings.

\section{Results}
Placeholder for quantitative and qualitative outcomes.

\section{Discussion}
We will interpret early findings with respect to workflow integration,
human--AI collaboration, multidisciplinary coordination, and design principles.
Limitations and future work will be detailed.

\section{Conclusion}
The framework provides a foundation for integrative, AI-enhanced encounters,
addressing gaps in EHR-based AI by emphasizing continuity, modularity, and
collaboration.

\section{Ethics}
All data used in sessions are synthetic clinical vignettes; no real patient
data are accessed. Clinicians provide informed consent. Ethics review will be
obtained as required for studies involving clinicians and simulated cases.

\section{References}
\bibliographystyle{unsrt}
\bibliography{references}

\end{document}


% === Appendix B: Physician Interaction Protocol (Baseline vs. HPH) ===
\section{Physician Interaction Protocol}
\label{app:interaction}

\subsection*{Purpose}
This appendix specifies exactly how clinicians interact with the baseline
(OpenMRS) and the HPH prototype to ensure parity, reproducibility, and
clarity for replication.

\subsection*{Systems under comparison}
\begin{itemize}
  \item \textbf{Baseline (Standard):} OpenMRS Reference Application with FHIR2
  (FHIR R4). No decision-support/AI modules enabled.
  \item \textbf{HPH (Prototype):} HPH UI reading the identical FHIR Bundle per
  case. Assistive features (summaries, rationale, provenance) are enabled and
  labeled as such.
\end{itemize}

\subsection*{Data seeding and interaction rule}
Both conditions use the \emph{identical} canonical FHIR R4 Bundle per case;
OpenMRS is pre-seeded via FHIR2 \emph{before} sessions, and HPH reads the same
Bundle. Participants interact only through the respective UIs. No manual data
entry or edits are performed during sessions.

\subsection*{Pre-session setup (facilitator checklist)}
\begin{enumerate}
  \item Verify the correct role-based account (IM/FP, EM, Psychiatry) is
        logged in for the participant.
  \item Open the assigned case (\emph{A} or \emph{B}) on the landing page
        of the designated system (Baseline or HPH) according to the
        counterbalanced order sheet.
  \item Confirm that Notes, Labs, Medications, Imaging/Reports, and Timeline
        views are accessible.
  \item Start screen and audio capture; confirm consent and think-aloud.
  \item Set a visible countdown timer for the block (10--12 minutes).
\end{enumerate}

\subsection*{Session design and timing}
Within-subjects, counterbalanced (Latin square: Case A/B $\times$ Baseline/HPH).
Each participant completes \textbf{both} cases in \textbf{both} workflows.
\begin{enumerate}
  \item Block~1: Case (A or B) with \emph{Workflow 1} (Baseline or HPH) (10--12 min).
  \item Block~2: Same Case with \emph{Workflow 2} (HPH or Baseline) (10--12 min).
  \item Repeat Blocks 1--2 with the other Case (B or A).
  \item Post-task questionnaires (Likert scales) and 8--10 min interview.
\end{enumerate}

\subsection*{Task checklist (display to participants)}
\begin{itemize}
  \item Identify immediate risks and red flags.
  \item Propose initial orders/actions (labs, imaging, meds) and disposition.
  \item Provide a short written rationale (2--3 lines) in the notes area.
  \item Identify information gaps or missing data that could change decisions.
\end{itemize}

\subsection*{Role prompts (shown on a card beside the screen)}
\begin{itemize}
  \item \textbf{IM/FP:} Coherent plan for first visit/urgent setting; continuity
        considerations for follow-up.
  \item \textbf{EM:} Time-sensitive safety and escalation; admission vs. DC.
  \item \textbf{Psych:} Anxiety/panic overlay, interactions, and longitudinal
        risks; how continuity artifacts assist judgment.
\end{itemize}

\subsection*{Baseline (OpenMRS) navigation map}
\begin{enumerate}
  \item \textbf{Landing:} Patient dashboard for the assigned case.
  \item \textbf{Notes:} Read triage and progress notes; add a brief rationale
        (if a notes area is provided for the study user).
  \item \textbf{Labs:} Review time-stamped labs and reference ranges.
  \item \textbf{Medications:} Review current/previous meds and adherence notes.
  \item \textbf{Imaging/Reports:} Read ECG/CXR text reports (no images).
  \item \textbf{Timeline:} Cross-check event order (onset $\rightarrow$ tests).
\end{enumerate}

\subsection*{HPH navigation map}
\begin{enumerate}
  \item \textbf{Overview:} Read structured summary and provenance panel.
  \item \textbf{Notes:} Read triage/progress excerpts with citations.
  \item \textbf{Labs:} Review trends/flags (values equal to Baseline).
  \item \textbf{Medications:} Review list; see interaction checks if present.
  \item \textbf{Reports:} Read ECG/CXR text; follow provided citations.
  \item \textbf{Timeline:} Verify event order and latency between steps.
\end{enumerate}

\subsection*{Interaction constraints (parity enforcement)}
\begin{itemize}
  \item No external search, no copy/paste into other tools.
  \item No new orders actually placed; all actions are hypothetical.
  \item No editing of patient data; notes are study artifacts only.
  \item Stay within the allocated time; notify when done.
\end{itemize}

\subsection*{Captured artifacts and measures}
\begin{itemize}
  \item Screen/audio recording with timestamps.
  \item Notes/rationales entered (study area only).
  \item Task completion indicators (checklist).
  \item Likert outcomes: usefulness, decision support, explainability,
        information continuity, workflow fit; optional NASA-TLX.
\end{itemize}

\subsection*{Moderator prompts (use sparingly)}
\begin{itemize}
  \item ``Please think aloud as you work.''
  \item ``What would you do next, and why?''
  \item ``What information is missing or uncertain?''
  \item ``How did this view help or hinder your decision?''
\end{itemize}

\subsection*{Deviations and fail-safes}
\begin{itemize}
  \item If a page fails to load, pause the timer, refresh once, then continue.
  \item If a role account logs out, resume with the same case and workflow.
  \item Record any deviation and its duration in the session log.
\end{itemize}

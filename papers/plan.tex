
\documentclass[12pt]{article}
\usepackage{jmiremu}
\usepackage[toc,page]{appendix} % in the preamble

\title{HPH Framework: Research Project Plan}
\author[1]{Ivan Ogasawara}
\affil[1]{Independent Researcher, IGDORE}
\date{November 2025}

\begin{document}
\maketitle

\section{Background and Rationale}
High-quality clinical care hinges on continuity and accessibility of
information across the entire encounter: screening, diagnosis, treatment,
prescription, and follow-up. Despite major gains in electronic health record
(EHR) adoption over the past decade, fragmentation and workflow misalignment
persist in practice. For example, U.S. office-based physician EHR adoption
reached 88.2\% in 2021 \cite{cdc_nehrs_2021}, and sector-specific settings
(e.g., assisted living communities) have also seen rapid growth in EHR and
health information exchange (HIE) functions \cite{lin2022jamia}. Pre-pandemic
assessments document a decade of progress (2009--2019), but also reveal gaps
in interoperability and use that impact team coordination \cite{jiang2023}.

Concurrently, artificial intelligence (AI)---including large language models
(LLMs), retrieval-augmented generation (RAG), and multimodal learning---is
increasingly applied to clinical decision support (CDS). Recent reviews
highlight both emerging utility and persistent adoption barriers, including
trust, explainability, and workflow fit \cite{tun2025trust,bmc_hf_2024,
simon2025mmai}. Most existing systems remain narrow, supporting a single step
(e.g., triage, risk prediction, imaging) rather than the longitudinal,
multidisciplinary workflow clinicians require.

\subsection*{Context: National deployments (DoctorSV)}
Recent national initiatives such as El Salvador’s DoctorSV signal momentum
toward unified digital health services and a ``single clinical history''
vision. Public communications describe phased onboarding and integration
with pharmacies, laboratories, and imaging providers. While independent
evaluations are pending, these programs underscore the need for modular,
standards-aligned, and auditable AI frameworks that can integrate with
national infrastructures \cite{caf2025doctorsv}.

\section{Significance and Innovation}
The project contributes to medical informatics by:
\begin{itemize}
  \item Integrating multimodal AI components across screening-to-follow-up
  within one longitudinal workflow.
  \item Embedding transparency, provenance, and human oversight into each AI
  interaction (audit trails, citations, model cards).
  \item Providing a plug-in architecture compatible with FHIR and open data
  models, enabling site-specific extensibility.
  \item Addressing trust, usability, and task--technology fit from the outset
  through mixed-methods evaluation with clinicians.
\end{itemize}

\subsection*{Implications for national platforms}
Although our evaluation uses synthetic cases in a research prototype,
the framework’s emphasis on FHIR compatibility, provenance/audit logs,
and modular AI suggests a natural path for integration with emerging
national platforms (e.g., DoctorSV), contingent on available interfaces,
governance agreements, and safety validation. We did not analyze or
evaluate any national deployment; references to such programs are purely
contextual.

\section{Objectives and Research Questions}
\textbf{Primary research question:} How can an AI-augmented, EHR-centered
framework improve information continuity and perceived decision support across
multidisciplinary care?

\noindent\textbf{Secondary research questions:}
\begin{enumerate}
  \item Which design principles enable effective multi-phase AI integration?
  \item How do clinicians perceive trust, explainability, usability, and
  workflow fit when using the framework?
  \item What technical and organizational challenges arise during early
  adoption in simulated settings?
\end{enumerate}

\section{Conceptual Framework}
We model the encounter as five interconnected stages:
\begin{enumerate}
  \item \textbf{Screening \& Triage:} symptom capture, patient-reported
  information, AI-assisted risk estimation.
  \item \textbf{Diagnostic Reasoning:} multimodal data integration and
  differential support with literature-backed rationale.
  \item \textbf{Treatment Planning:} guideline-aware plan synthesis and
  safety checks (e.g., interactions).
  \item \textbf{Prescription \& Care Execution:} structured orders and care
  plans documented within the unified record.
  \item \textbf{Follow-up \& Prognosis:} monitoring, prediction, and
  longitudinal review.
\end{enumerate}
A unified longitudinal record maintains structured fields (labs, vitals,
medications), unstructured notes, and AI-generated artifacts with provenance.

\section{Methodology}
\subsection{Study Design}
A mixed-methods, within-subjects, counterbalanced study combining iterative
system development with early evaluation using synthetic cases and clinician
participants.

\subsection{Baseline Workflow (Standard): OpenMRS Reference Application}
The baseline condition uses the OpenMRS Reference Application (RefApp) with
the FHIR2 module (FHIR R4). We will load the same synthetic cases used in the
HPH condition into OpenMRS via FHIR Bundle POSTs. Baseline sessions disable
any decision-support or AI add-ons. Participants interact with standard
OpenMRS views for: Notes, Labs, Medications, Imaging/Reports, and a patient
timeline. Role-based accounts (IM/FP, EM, Psychiatry) will have read/enter
permissions appropriate for data review and note entry.

\subsection{HPH Workflow (Prototype)}
The HPH condition uses our prototype UI backed by the identical FHIR Bundles.
The prototype renders the same clinical content and adds assistive features
(e.g., structured summaries, literature-backed suggestions, provenance). The
prototype does not expose information unavailable in the baseline.

\subsection{Parity Rules (Pre-registered)}
We enforce parity to isolate the effect of the framework:
\begin{itemize}
  \item \textbf{Seeding and interaction:} Both conditions use the identical
  canonical FHIR R4 Bundle per case; OpenMRS is pre-seeded via FHIR2
  (pre-session), and the HPH prototype reads the same Bundle. Clinicians
  interact only through the respective UIs; no data are entered during sessions.
  \item \textbf{Content parity:} identical synthetic cases, timestamps, and
  missingness patterns in both conditions.
  \item \textbf{Task parity:} identical task lists (e.g., initial orders, risk
  estimate, disposition).
  \item \textbf{Time parity:} matched time windows per block.
  \item \textbf{Feature parity:} no decision-support in baseline; assistive
  features in HPH are labeled and auditable.
\end{itemize}

\subsection{Synthetic Case Generation and Artifact Pipeline}
Two EHR-formatted vignettes (acute and chronic multimorbidity) will be
constructed from guidelines/textbooks and refined by 2--3 clinicians. Each
case is authored as a canonical \textbf{FHIR R4 Bundle} (Patient, Encounter,
Observation for labs/vitals, MedicationStatement/Request, DiagnosticReport for
ECG/CXR text, ImagingStudy metadata, AllergyIntolerance, CarePlan,
Composition). From the Bundle we derive UI artifacts for both conditions:
\begin{itemize}
  \item \textbf{OpenMRS baseline:} create records via FHIR POSTs; use native
  OpenMRS views for review.
  \item \textbf{HPH prototype:} read the same Bundle to render equivalent
  views plus assistive summaries.
\end{itemize}
Each case folder contains a manifest (hashes, counts of resources) and frozen
version to ensure reproducibility.

\subsection{Participants}
We will recruit 6--12 physicians across specialties (e.g., internal medicine,
family medicine, emergency, psychiatry). Inclusion criteria: licensed MD, $\geq$
1 year of routine EHR use. Exclusion: prior contribution to this study's
prototype design beyond informal feedback.

\subsection{Session Flow and Counterbalancing}
Each participant completes both cases under both workflows. Order is
counterbalanced across participants (Latin square: Case A/B $\times$ Baseline/HPH).
A typical session (40--60 minutes): consent and orientation; Block~1 (case~$\times$
workflow); Block~2 (the alternate workflow on the same case); repeat for the
second case; questionnaires and short interview.

\subsection{Measures}
\textbf{Quantitative (5-point Likert):} perceived usefulness, decision-support
quality, information continuity, explainability, workflow fit; optional
NASA-TLX workload. \textbf{Qualitative:} interview themes on strengths,
limitations, collaborative use, and safety.

\subsection{Analysis}
Wilcoxon signed-rank tests (and effect sizes) compare framework-assisted
versus baseline workflows within-subjects. Thematic analysis (inductive
coding) is applied to interview transcripts. Mixed-methods triangulation
integrates quantitative and qualitative findings.

\section{Ethical Considerations}
No real patient data will be collected or analyzed. Vignettes are synthetic
and de-identified. Clinicians (the only human participants) will provide
informed consent. The protocol will be submitted to an institutional ethics
committee to confirm compliance with regulations for studies involving
clinicians and simulated cases.

\section{Risk Management}
Risks include data misuse, overreliance on AI suggestions, and participant
fatigue. Mitigations: strict separation from real EHRs; clear labeling of AI
outputs as assistive; capped session length; and audit trails for all AI
artifacts.

\section{Timeline}
\begin{table}[H]
\centering
\begin{tabular}{ll}
\toprule
\textbf{Phase} & \textbf{Duration \& Key Tasks} \\
\midrule
Design \& Requirements & Months 1--3: literature review; architecture; IRB prep. \\
Prototype Development & Months 4--6: build plugins; unify record; interface. \\
Evaluation & Months 7--8: clinician sessions; data collection. \\
Analysis \& Writing & Months 9--10: analysis; manuscript; preregistration. \\
\bottomrule
\end{tabular}
\end{table}

\section{Dissemination Plan}
We will prepare a preprint and target a digital health venue (e.g., JMIR AI,
JAMIA Open, Frontiers in Digital Health). Open-source components that do not
process patient data will be released under a permissive license, with
documentation and demo notebooks.

\section{Code and Data Availability}
All non-proprietary software components and study protocols will be released
under a permissive open-source license (e.g., BSD-3-Clause). A code
availability statement will be included as recommended by JMIR
\cite{jmir_data_policy2025}, with repository URL and an archived DOI
(immutable release) cited in the manuscript.

\section{Expected Outcomes}
\begin{itemize}
  \item A validated conceptual and technical framework for AI-integrated
  encounters.
  \item A working prototype demonstrating cross-phase AI interoperability.
  \item Empirical insights into clinician trust, explainability, and workflow
  integration in simulated studies.
\end{itemize}

\section{References}
\bibliographystyle{unsrt}
\bibliography{references}

\clearpage
\begin{appendices}
  
% === Appendix: Synthetic Vignette + Administration Protocol ===

\section{Synthetic Vignette and Administration Protocol}
\label{app:vignette}

\subsection*{Vignette A (Acute Care: Chest Pain in Primary/Urgent Setting)}
\textbf{Chief complaint:} ``Chest pain for 3 hours.''

\textbf{Triage note (nurse, arrival T0):}
\begin{itemize}
  \item 54-year-old male; abrupt retrosternal pressure while lifting boxes.
  \item Pain 7/10, radiates to left arm; associated diaphoresis and mild nausea.
  \item No syncope; no dyspnea at rest. Walk-in; arrived by car.
\end{itemize}

\textbf{Past medical history:}
\begin{itemize}
  \item Hypertension (10 years), hyperlipidemia, former smoker (20 pack-years).
  \item Father died of MI at 58.
\end{itemize}

\textbf{Medications/allergies:}
\begin{itemize}
  \item Amlodipine 10 mg daily; Atorvastatin 40 mg nightly.
  \item No known drug allergies.
\end{itemize}

\textbf{Vital signs at T0:}
\begin{itemize}
  \item BP 156/92 mmHg; HR 104 bpm; RR 18; Temp 36.8$^\circ$C; SpO$_2$ 97\% RA.
\end{itemize}

\textbf{Initial tests (available at T+15 min):}
\begin{itemize}
  \item ECG: sinus tachycardia; non-specific ST-T changes (no ST elevation).
  \item Troponin I (high-sensitivity): pending at T+15; available at T+60.
  \item Basic labs (CBC, BMP, lipid panel): pending.
\end{itemize}

\textbf{Imaging (available on request):}
\begin{itemize}
  \item Portable chest X-ray: not yet performed.
\end{itemize}

\textbf{Progress note (physician, T+60):}
\begin{itemize}
  \item Troponin I (hs): 24 ng/L (ULN 18); repeat at T+120 pending.
  \item Pain decreased to 3/10 after rest; persists with exertion.
\end{itemize}

\textbf{Uncertainty/missingness intentionally included:}
\begin{itemize}
  \item No prior ECG on file; medication adherence unclear.
  \item Exact symptom onset time self-reported; no external records available.
\end{itemize}

\textbf{Artifacts to load in the prototype:}
\begin{itemize}
  \item Triage note, vital trend table, structured meds list, ECG text report,
        labs table (with timestamps), imaging placeholder, timeline view.
\end{itemize}

\subsection*{Role-based prompts (to display in-session)}
\begin{itemize}
  \item \textbf{Generalist (IM/FP):} Identify immediate risks, initial orders,
        and disposition. How does the framework support information continuity?
  \item \textbf{Emergency:} Triage safety, time-sensitive decisions, and
        escalation criteria. What signals are surfaced at a glance?
  \item \textbf{Psychiatry/Mental Health:} Screen for anxiety/panic overlay,
        medication interactions, and follow-up continuity. Which longitudinal
        elements matter to you here?
\end{itemize}

\subsection*{Administration protocol (within-subjects, counterbalanced)}
\begin{enumerate}
  \item \textbf{Design:} 2 (workflow: standard vs. prototype) $\times$ 2 (case:
        A vs. B) within-subjects; Latin-square counterbalancing of order.
  \item \textbf{Flow per participant (40--60 min):}
    \begin{enumerate}
      \item Consent; 2-min orientation; start think-aloud.
      \item Block 1: Case (A or B) with \emph{standard workflow}. (10--12 min)
      \item Block 2: Case (A or B) with \emph{prototype}. (10--12 min)
      \item Repeat Blocks 1--2 with the other case, swapping workflow order.
      \item Post-task questionnaires (Likert scales) and 8--10 min interview.
    \end{enumerate}
  \item \textbf{Measures:} usefulness, decision support, explainability,
        information continuity, workflow fit, optional NASA-TLX.
  \item \textbf{Artifacts captured:} screen/audio recording, timestamps, orders
        placed, notes generated (de-identified), questionnaire responses.
\end{enumerate}

\subsection*{Sample size rationale}
With a within-subjects design, each clinician experiences both workflows on
both cases, allowing paired comparisons. Usability and early mixed-methods
studies commonly use $n=6$--$12$ to identify the majority of major issues and
reach thematic saturation across roles while keeping sessions feasible.
Stratifying specialties (e.g., IM/FP, EM, Psychiatry) ensures role-specific
feedback.

\subsection*{Notes}
Case B (chronic, multimorbidity) is prepared similarly (e.g., type 2 diabetes,
hypertension, depression, polypharmacy) with longitudinal labs, meds changes,
and missingness (e.g., gaps in adherence). This appendix illustrates structure
and administration; all values are synthetic and for research use.

  \clearpage
  
% === Appendix B: Physician Interaction Protocol (Baseline vs. HPH) ===
\section{Physician Interaction Protocol}
\label{app:interaction}

\subsection*{Purpose}
This appendix specifies exactly how clinicians interact with the baseline
(OpenMRS) and the HPH prototype to ensure parity, reproducibility, and
clarity for replication.

\subsection*{Systems under comparison}
\begin{itemize}
  \item \textbf{Baseline (Standard):} OpenMRS Reference Application with FHIR2
  (FHIR R4). No decision-support/AI modules enabled.
  \item \textbf{HPH (Prototype):} HPH UI reading the identical FHIR Bundle per
  case. Assistive features (summaries, rationale, provenance) are enabled and
  labeled as such.
\end{itemize}

\subsection*{Data seeding and interaction rule}
Both conditions use the \emph{identical} canonical FHIR R4 Bundle per case;
OpenMRS is pre-seeded via FHIR2 \emph{before} sessions, and HPH reads the same
Bundle. Participants interact only through the respective UIs. No manual data
entry or edits are performed during sessions.

\subsection*{Pre-session setup (facilitator checklist)}
\begin{enumerate}
  \item Verify the correct role-based account (IM/FP, EM, Psychiatry) is
        logged in for the participant.
  \item Open the assigned case (\emph{A} or \emph{B}) on the landing page
        of the designated system (Baseline or HPH) according to the
        counterbalanced order sheet.
  \item Confirm that Notes, Labs, Medications, Imaging/Reports, and Timeline
        views are accessible.
  \item Start screen and audio capture; confirm consent and think-aloud.
  \item Set a visible countdown timer for the block (10--12 minutes).
\end{enumerate}

\subsection*{Session design and timing}
Within-subjects, counterbalanced (Latin square: Case A/B $\times$ Baseline/HPH).
Each participant completes \textbf{both} cases in \textbf{both} workflows.
\begin{enumerate}
  \item Block~1: Case (A or B) with \emph{Workflow 1} (Baseline or HPH) (10--12 min).
  \item Block~2: Same Case with \emph{Workflow 2} (HPH or Baseline) (10--12 min).
  \item Repeat Blocks 1--2 with the other Case (B or A).
  \item Post-task questionnaires (Likert scales) and 8--10 min interview.
\end{enumerate}

\subsection*{Task checklist (display to participants)}
\begin{itemize}
  \item Identify immediate risks and red flags.
  \item Propose initial orders/actions (labs, imaging, meds) and disposition.
  \item Provide a short written rationale (2--3 lines) in the notes area.
  \item Identify information gaps or missing data that could change decisions.
\end{itemize}

\subsection*{Role prompts (shown on a card beside the screen)}
\begin{itemize}
  \item \textbf{IM/FP:} Coherent plan for first visit/urgent setting; continuity
        considerations for follow-up.
  \item \textbf{EM:} Time-sensitive safety and escalation; admission vs. DC.
  \item \textbf{Psych:} Anxiety/panic overlay, interactions, and longitudinal
        risks; how continuity artifacts assist judgment.
\end{itemize}

\subsection*{Baseline (OpenMRS) navigation map}
\begin{enumerate}
  \item \textbf{Landing:} Patient dashboard for the assigned case.
  \item \textbf{Notes:} Read triage and progress notes; add a brief rationale
        (if a notes area is provided for the study user).
  \item \textbf{Labs:} Review time-stamped labs and reference ranges.
  \item \textbf{Medications:} Review current/previous meds and adherence notes.
  \item \textbf{Imaging/Reports:} Read ECG/CXR text reports (no images).
  \item \textbf{Timeline:} Cross-check event order (onset $\rightarrow$ tests).
\end{enumerate}

\subsection*{HPH navigation map}
\begin{enumerate}
  \item \textbf{Overview:} Read structured summary and provenance panel.
  \item \textbf{Notes:} Read triage/progress excerpts with citations.
  \item \textbf{Labs:} Review trends/flags (values equal to Baseline).
  \item \textbf{Medications:} Review list; see interaction checks if present.
  \item \textbf{Reports:} Read ECG/CXR text; follow provided citations.
  \item \textbf{Timeline:} Verify event order and latency between steps.
\end{enumerate}

\subsection*{Interaction constraints (parity enforcement)}
\begin{itemize}
  \item No external search, no copy/paste into other tools.
  \item No new orders actually placed; all actions are hypothetical.
  \item No editing of patient data; notes are study artifacts only.
  \item Stay within the allocated time; notify when done.
\end{itemize}

\subsection*{Captured artifacts and measures}
\begin{itemize}
  \item Screen/audio recording with timestamps.
  \item Notes/rationales entered (study area only).
  \item Task completion indicators (checklist).
  \item Likert outcomes: usefulness, decision support, explainability,
        information continuity, workflow fit; optional NASA-TLX.
\end{itemize}

\subsection*{Moderator prompts (use sparingly)}
\begin{itemize}
  \item ``Please think aloud as you work.''
  \item ``What would you do next, and why?''
  \item ``What information is missing or uncertain?''
  \item ``How did this view help or hinder your decision?''
\end{itemize}

\subsection*{Deviations and fail-safes}
\begin{itemize}
  \item If a page fails to load, pause the timer, refresh once, then continue.
  \item If a role account logs out, resume with the same case and workflow.
  \item Record any deviation and its duration in the session log.
\end{itemize}

\end{appendices}

\end{document}


% === Appendix: Synthetic Vignette + Administration Protocol ===

\section{Synthetic Vignette and Administration Protocol}
\label{app:vignette}

\subsection*{Vignette A (Acute Care: Chest Pain in Primary/Urgent Setting)}
\textbf{Chief complaint:} ``Chest pain for 3 hours.''

\textbf{Triage note (nurse, arrival T0):}
\begin{itemize}
  \item 54-year-old male; abrupt retrosternal pressure while lifting boxes.
  \item Pain 7/10, radiates to left arm; associated diaphoresis and mild nausea.
  \item No syncope; no dyspnea at rest. Walk-in; arrived by car.
\end{itemize}

\textbf{Past medical history:}
\begin{itemize}
  \item Hypertension (10 years), hyperlipidemia, former smoker (20 pack-years).
  \item Father died of MI at 58.
\end{itemize}

\textbf{Medications/allergies:}
\begin{itemize}
  \item Amlodipine 10 mg daily; Atorvastatin 40 mg nightly.
  \item No known drug allergies.
\end{itemize}

\textbf{Vital signs at T0:}
\begin{itemize}
  \item BP 156/92 mmHg; HR 104 bpm; RR 18; Temp 36.8$^\circ$C; SpO$_2$ 97\% RA.
\end{itemize}

\textbf{Initial tests (available at T+15 min):}
\begin{itemize}
  \item ECG: sinus tachycardia; non-specific ST-T changes (no ST elevation).
  \item Troponin I (high-sensitivity): pending at T+15; available at T+60.
  \item Basic labs (CBC, BMP, lipid panel): pending.
\end{itemize}

\textbf{Imaging (available on request):}
\begin{itemize}
  \item Portable chest X-ray: not yet performed.
\end{itemize}

\textbf{Progress note (physician, T+60):}
\begin{itemize}
  \item Troponin I (hs): 24 ng/L (ULN 18); repeat at T+120 pending.
  \item Pain decreased to 3/10 after rest; persists with exertion.
\end{itemize}

\textbf{Uncertainty/missingness intentionally included:}
\begin{itemize}
  \item No prior ECG on file; medication adherence unclear.
  \item Exact symptom onset time self-reported; no external records available.
\end{itemize}

\textbf{Artifacts to load in the prototype:}
\begin{itemize}
  \item Triage note, vital trend table, structured meds list, ECG text report,
        labs table (with timestamps), imaging placeholder, timeline view.
\end{itemize}

\subsection*{Role-based prompts (to display in-session)}
\begin{itemize}
  \item \textbf{Generalist (IM/FP):} Identify immediate risks, initial orders,
        and disposition. How does the framework support information continuity?
  \item \textbf{Emergency:} Triage safety, time-sensitive decisions, and
        escalation criteria. What signals are surfaced at a glance?
  \item \textbf{Psychiatry/Mental Health:} Screen for anxiety/panic overlay,
        medication interactions, and follow-up continuity. Which longitudinal
        elements matter to you here?
\end{itemize}

\subsection*{Administration protocol (within-subjects, counterbalanced)}
\begin{enumerate}
  \item \textbf{Design:} 2 (workflow: standard vs. prototype) $\times$ 2 (case:
        A vs. B) within-subjects; Latin-square counterbalancing of order.
  \item \textbf{Flow per participant (40--60 min):}
    \begin{enumerate}
      \item Consent; 2-min orientation; start think-aloud.
      \item Block 1: Case (A or B) with \emph{standard workflow}. (10--12 min)
      \item Block 2: Case (A or B) with \emph{prototype}. (10--12 min)
      \item Repeat Blocks 1--2 with the other case, swapping workflow order.
      \item Post-task questionnaires (Likert scales) and 8--10 min interview.
    \end{enumerate}
  \item \textbf{Measures:} usefulness, decision support, explainability,
        information continuity, workflow fit, optional NASA-TLX.
  \item \textbf{Artifacts captured:} screen/audio recording, timestamps, orders
        placed, notes generated (de-identified), questionnaire responses.
\end{enumerate}

\subsection*{Sample size rationale}
With a within-subjects design, each clinician experiences both workflows on
both cases, allowing paired comparisons. Usability and early mixed-methods
studies commonly use $n=6$--$12$ to identify the majority of major issues and
reach thematic saturation across roles while keeping sessions feasible.
Stratifying specialties (e.g., IM/FP, EM, Psychiatry) ensures role-specific
feedback.

\subsection*{Notes}
Case B (chronic, multimorbidity) is prepared similarly (e.g., type 2 diabetes,
hypertension, depression, polypharmacy) with longitudinal labs, meds changes,
and missingness (e.g., gaps in adherence). This appendix illustrates structure
and administration; all values are synthetic and for research use.

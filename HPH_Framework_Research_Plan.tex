
\documentclass[12pt]{article}
\usepackage{jmiremu}

\title{HPH Framework: Research Project Plan}
\author[1]{Sandro Loch}
\affil[1]{Independent Researcher, IGDORE}
\date{November 2025}

\begin{document}
\maketitle

\section{Background and Rationale}
Delivering safe and effective care requires information continuity across
screening, diagnosis, treatment, and follow-up. Many EHR deployments remain
fragmented or task-specific. The HPH Framework proposes an AI-augmented,
EHR-centered system that integrates all stages of the medical encounter and
supports multidisciplinary collaboration.

\section{Objectives and Research Questions}
\textbf{Primary research question:} How can an AI-augmented, EHR-centered
framework improve information continuity and perceived decision support across
multidisciplinary care?

\noindent\textbf{Secondary questions:}
\begin{enumerate}
  \item Which design principles enable multi-phase AI integration?
  \item How do clinicians perceive trust, usability, and workflow fit?
\end{enumerate}

\section{Conceptual Framework Overview}
The encounter is modeled as five stages: screening, diagnosis, treatment
planning, prescription, and follow-up. Each stage is supported by modular AI
components within a unified longitudinal patient record.

\section{Methodology}
A mixed-methods exploratory study will combine system design and clinician
evaluation using synthetic clinical vignettes. Participants will interact with
a prototype and report usability, decision support, and information flow.

\section{Ethical Considerations}
All cases are synthetic; no patient data are accessed. Clinicians who evaluate
the prototype will provide informed consent. Ethics review will be obtained
as required for studies involving health professionals.

\section{Timeline}
Phase 1 (Design): 3 months. Phase 2 (Prototype): 3 months. Phase 3
(Evaluation): 2 months. Phase 4 (Analysis and Writing): 2 months.

\section{Expected Outcomes}
A validated conceptual framework, a working prototype, and a manuscript
describing design and early evaluation results.

\end{document}
